\documentclass[12pt]{amsart}
\usepackage{todonotes}
\usepackage{amssymb}

\newtheorem{thm}{Theorem}
\newtheorem{prop}{Proposition}
\newtheorem{lem}{Lemma}
\newtheorem{cor}{Corollary}

\title{A Lazy person's approach to wavelet theory on manifolds}
\author{Henry O. Jacobs and Ram Vasudevan}
\date{18th of August, 2014}

\begin{document}
\maketitle

Let $M$ be a manifold.
We call a chart $\varphi:V \subset \mathbb{R}^n \to U \subset M$ a
\emph{full-chart} if $U$ is dense in $M$.


\begin{prop}
If $M$ is compact then there exists a full chart.
\end{prop}
\begin{proof}
Without loss of generality we will assume that $M$ is connected.
If $M$ is not connected we may apply the following argument to each component.
Equip $M$ with a Riemannian structure.
We may then invoke Lemma 4.4 of \cite{Sakai1996}.
This completes the proof, but we can provide details for the sake of 
completeness (pun intended).
For any $p \in M$ we can define the cut-locus by $C_p$ and the maximal open set $U_p \subset T_p M$ over which $\exp_p : T_p M \to M$ is injective.
We see that $U = \exp_p(U_p)$ is dense in $M$.
Upon choosing a basis, ${\bf e}_1,\dots,{\bf e}_n \in T_xM$,
 we obtain a chart $\varphi: V \subset \mathbb{R}^n \to U \subset M$.
Here $V$ is a star shaped subset of $\mathbb{R}^n$.
In particular, $\varphi(x) = \exp_p ( x^i {\bf e}_i )$.
\end{proof}

Given a full-chart, $\varphi$, we can extend it to a surjective map $\bar{\varphi} : \bar{V} \to M$
such that $\bar{\varphi}(x) = \varphi(x)$ when $x \in V$ and $\bar{\varphi}(x) = \lim_{i \to \infty} \varphi(x_i)$ if $x \in \partial V$ for an arbitrary sequence $\{ x_i \} \subset V$ such that $\lim_{i \to \infty} x_i = x$.
The map $\bar{\varphi}$ is injective on $V$, but it is generally not injective on the boundary $\partial V$.
Thus the equivalence relation,
 $x \sim y \iff \bar{\varphi}(x) = \bar{\varphi}(y)$,
 is typically not a simple equality.

\begin{prop} \label{prop:manifold}
  Given a manifold $M$ and a full chart $\varphi:V \to U$, the space $\bar{V} / \bar{\varphi}$ is diffeomorphic to $M$
\end{prop}
\begin{proof}
  By construction $\bar{\varphi}$ is constant on each equivalence class,
and thus induces a map from $\bar{V} /\bar{\varphi} \to M$.
It is simple to verify this map is bijective.
Thus $M$ and $\bar{V} / \bar{\varphi}$ are homeomorphic under the quotient topology.
  In particular, the map $\bar{\varphi}$ creates a bijection between open sets in $M$ and open sets (in the quotient topology) on $\bar{V}$.
  Thus $\bar{V}$ inherits the differential structure of $M$.
\end{proof}

\begin{cor} \label{cor:functions}
  The ring of real-valued functions $C^k(M)$ is isomorphic to the subring
  $\bar{\varphi}^* C^k(M) := \{ f \circ \bar{\varphi} \mid f \in C^k(M) \}
  \subset C^k( \bar{V})$.
\end{cor}

The above proposition and its corollary are useful for our purposes because
they bring functional analysis on a manifold $M$ to functional analysis on a subring of functions in a Euclidean domain.
As much of the infrastructure for wavelets is built on Euclidean domains, we can hope to use these observations to use the infrastructure with only
minor modification.
This is most clearly described in the following examples.

\subsection{The unit circle}
Consider the unit-circle $S^1 = \{ e^{i \theta} \mid \theta \in [-\pi,\pi] \}$,
with the full chart $\varphi( \theta ) = e^{i\theta}$ for $\theta \in (-\pi,\pi)$, obtained by choosing some branch of the logarithm map.
The extension to $[-\pi,\pi]$ is obvious, and we can easily observe that $\bar{\varphi}(\pi) = \bar{\varphi}(-\pi)$, so that boundary points have been identified.
If we denote the equivalence of $\pi$ with $-\pi$ by $\sim$, then proposition \ref{prop:manifold} says that $S^1$ is identical to $[-\pi,\pi] / \sim$.
We also find $C^0(S^1)$ is identical to functions $f \in C^0( [\pi,\pi])$ such that $f(\pi) = f(-\pi)$.
The space $C^1(S^1)$ is identical to functions $f \in C^1( [-\pi,\pi])$ such that $f(\pi) = f(-\pi)$ and $f'(\pi) = f'(-\pi)$.
 We can construct wavelet basis for $S^1$ by wrapping functions on $[-\pi,\pi]$.


\subsection{Spheres}
Lets begin with $S^2$ using spherical coordinates $(\theta , \phi)$ on the open rectangle $V = (-\pi,\pi) \times (0,\pi)$.
We can consider the full-chart
\begin{align*}
  \varphi(\theta,\phi) = \begin{pmatrix}
    \cos(\theta) \sin(\phi) \\
    \sin(\theta) \sin(\phi) \\
    \cos(\phi)
    \end{pmatrix}
\end{align*}
We see that the north and south poles are identified as single points because  $\bar{\varphi}( \theta , 0)$ is constant, as is $\bar{\varphi}(\theta, \pi)$.  Similarly, $\bar{\varphi}( \pi , \phi) = \bar{\varphi}( -\pi, \phi)$ for all $\phi \in (0,\pi)$.
Let $W(S^1)$ be a wavelet basis for $S^1$ and let $W([0,\pi])$ be a wavelet basis for $[0,\pi]$ with vanishing boundary conditions.
A $C^0$-wavelet basis for $S^2$ is given by
\begin{align*}
  W(S^2) = W(S^1) \otimes W([0,\pi]) + 1_{S^2} + \phi.
\end{align*}
To make something smoother is more involved, but can be done.

Virtually the same construction using $3$-dimensional spherical coordinates would allow us to handle the $3$-sphere.

\subsection{Projective spaces}

% In particular, the subring $\bar{\varphi}^*C^k(M)$ can be viewed as a direct sum of two space, the first of which is a standard space handled by wavelets.

% \begin{prop}
%   Let $M$ be a manifold with a full-chart, $\varphi:V \to U$
%   and let $f \in C^0( \partial V) \mapsto f^\uparrow \in C^0( \bar{V})$
%   be an extension operator.
%   Then any $f \in \bar{\varphi}^*C^0(M)$ can be uniquely decomposed as
%   \begin{align*}
%     f = f_V + f^{\uparrow}_{\partial V}
%   \end{align*}
%   Where $f \in C^0( \bar{V})$ vanishes at the boundary,
%   and $f_{\partial V} \in C^0( \partial V)$ is constant on level-sets of $\bar{\varphi}$.
% \end{prop}

% To describe $C^1$-functions in this way requires a little more care.
% One route is to describe the tangent bundle $TM$ first,
% and define $C^1$-functions by how they act on vectors.

\bibliographystyle{amsalpha}
\bibliography{/Users/hoj201/Dropbox/hoj_2014.bib}

\end{document}